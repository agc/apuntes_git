\documentclass[]{article}
\usepackage{fontspec}

\title{}
\author{}





\begin{document}
\section{Creación de un repositorio.}

Al usar init se crea un subdirectorio, denominado .git, que contiene los archivos del repositorio. Mediante \texttt{add} se hace el \texttt{tracking} de los archivos que se incluiran en el commit






\begin{verbatim}
  $ git init
  $ git add *.c
  $ git add README
  $ git commit -m 'initial project version'
\end{verbatim}

Para hacer una copia de un repositorio ya existente

\begin{verbatim}
 $ git clone git://github.com/schacon/grit.git mygrit
\end{verbatim}

el último parámetro es optativo, indica el nombre del directorio que se creará. Si no se proporciona, se le asignará el nombre git

La orden anterior crea un directorio llamado mygrit, inicializa un directorio .git dentro del él, vuelca todo los datos para este repositorio, y checkout una \texttt{working copy} de la última versión.


\end{document}
