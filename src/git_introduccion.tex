\documentclass[]{article}
\usepackage{fontspec}

\title{}
\author{}





\begin{document}
\section{Introducción. Conceptos importantes.}

Los archivos en Git se encuentran en  tres estados importantes: 
\begin{itemize}
	\item los archivos pueden estar \textit{commited} si están archivados en la base de datos
	\item \textit{modificados} cuando han cambiado pero no han sido todavía commited
	\item  \textit{staged}, cuando un archivo modificado ha sido señalado para ir en la siguiente instantanea commit
\end{itemize}

  


El \textit{working directory} es un volcado \textit{checkout} de una versión del  proyecto: los archivos se extraen de la versión comprimida de la base de datos y se colocan en el disco para ser usados o modificados

La \textit{staging area} es un archivo sencillo, contenido en el directorio git, que almacena información sobre lo que se incluirá en el próximo commit


\end{document}
