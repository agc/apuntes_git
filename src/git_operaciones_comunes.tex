\documentclass[]{article}
\usepackage{fontspec}

\title{}
\author{}





\begin{document}
\section*{Operaciones comunes.}
\begin{verbatim}
$ git status
# On branch master
# Untracked files:
#   (use "git add <file>..." to include in what will be committed)
#
#	README
nothing added to commit but untracked files present (use "git add" to track)
\end{verbatim}

para añadir al índice un archivo, marcándolo para que forme parte del próximo \texttt{commit (tracking)
}
\begin{verbatim}
git add README
\end{verbatim}

Si un archivo que ha sido previamente añadido a la\textit{staging area} es modificado debe ejecutarse el comando \verb= git add <archivo>= de nuevo para \textit{stage} la nueva versión del archivo 

\subsection*{Ejemplo de archivo .gitignore}
\begin{verbatim}
# a comment - this is ignored
*.a       # no .a files
!lib.a    # but do track lib.a, even though you're ignoring .a files above
/TODO     # only ignore the root TODO file, not subdir/TODO
build/    # ignore all files in the build/ directory
doc/*.txt # ignore doc/notes.txt, but not doc/server/arch.txt
\end{verbatim}

vemos el uso de carácteres comodin, los comentarios y el uso de ! para la negación

\subsection*{diff}

\verb= git diff= muestra las diferencias entre lo que hay en el \texttt{working directory} y lo que hay en el \texttt{staging area}. Por otra parte \texttt{git diff --cached} muestra lo que está \textit{staged}, lo que irá en el próximo \textit{commit} 

\section*{"Commiting" los cambios}


\verb=git commit -m "Story 182: Fix benchmarks for speed"=

\end{document}
