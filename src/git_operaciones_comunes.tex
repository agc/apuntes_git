\documentclass[]{article}
\usepackage{fontspec}

\title{}
\author{}





\begin{document}
\section*{Operaciones comunes.}
\begin{verbatim}
$ git status
# On branch master
# Untracked files:
#   (use "git add <file>..." to include in what will be committed)
#
#	README
nothing added to commit but untracked files present (use "git add" to track)
\end{verbatim}

para añadir al índice un archivo, marcándolo para que forme parte del próximo \texttt{commit (tracking)
}
\begin{verbatim}
git add README
\end{verbatim}

Si un archivo que ha sido previamente añadido a la\textit{staging area} es modificado debe ejecutarse el comando \verb= git add <archivo>= de nuevo para \textit{stage} la nueva versión del archivo 

\subsection*{Ejemplo de archivo .gitignore}
\begin{verbatim}
# a comment - this is ignored
*.a       # no .a files
!lib.a    # but do track lib.a, even though you're ignoring .a files above
/TODO     # only ignore the root TODO file, not subdir/TODO
build/    # ignore all files in the build/ directory
doc/*.txt # ignore doc/notes.txt, but not doc/server/arch.txt
\end{verbatim}

vemos el uso de carácteres comodin, los comentarios y el uso de ! para la negación

\subsection*{diff}

\verb= git diff= muestra las diferencias entre lo que hay en el \texttt{working directory} y lo que hay en el \texttt{staging area}. Por otra parte \texttt{git diff --cached} muestra lo que está \textit{staged}, lo que irá en el próximo \textit{commit} 

\section*{"Commiting" los cambios}


\verb=git commit -m "Story 182: Fix benchmarks for speed"=


Para evitar añadir previamente a \textit{stage area} los archivos \textbf{modificados} que ya se habían añadido previamente  se usa la opción \texttt{-a} 

\verb= git commit -a -m 'added new benchmarks' =

\subsection*{Eliminar archivos}

Si se borra un archivo \verb= rm leeme.txt = cuando se invoca \verb= git status = se mostrará como modificado pero no \textit{staged} Para que git se olvide del archivo y deje de seguirle la pista, se usa el comando \texttt{rm} \verb= git rm leeme.txt = 

Se puede usar sólamente el segundo comando, sin borrar previamente el archivo: git lo eliminará de la \textit{working area}

Se puede renombrar un archivo mediante \verb= git mv file_from file_to =

Para borrar un archivo del repositorio, pero no del sistema de ficheros \verb= git rm --cached mylogfile.log =

\section{Revisando el historial de commits}

\verb= git log -p -2 = muestra las diferencias introducidas en cada commit. \texttt{-2} se usa para limitar la salida a dos lineas

\verb= git log --stat = muestra algunas estadísticas abreivada para cada commit. Otra opción interesante es \texttt{--pretty}

Otra opción interesante es \texttt{format} (ver pág 29 de Pro git)

Para limitar el número de salidas \verb+ git log --since=2.weeks + Otras opciones

\begin{tabular}{ll}
	\textbf{Opción} & \textbf{Descripción}\\
	-(n) & Muestra sólo los n últimos commits\\
	--since, --after & Muestra sólo los hechos después de una fecha\\
	--until, --before & los hechos después de una fecha\\
	-- author & los hechos por el autor\\
	-- committer & los hechos por el committer\\
\end{tabular}

Un comando para mostrar el log \verb= git log --pretty=oneline =
\end{document}
