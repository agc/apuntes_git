\documentclass[]{article}
\usepackage{fontspec}



\begin{document}






\section{Tagging}

Para crear un tagg anotado, un tag ligero, hacer un listado de los tags, ver información de lose existentes, 
\begin{verbatim}
 git tag -a v1.4 -m 'my version 1.4'
 git tag
 git show v1.4
\end{verbatim}

Para crear un tag ligero se usa el comando \verb= git tag v1.4-lw = ; se observa que no se ha incluido ninguna opción \verb= -a -n =

Para hacer tag a un \texttt{commit} anterior se usa \verb= git tag -a v1.2 9fceb02 = donde las cifras hexadecimales son el comienzo del commit checksum ( se puede obtener con \verb= git log --pretty=oneline =)

Para transferir los tags al repositorio remoto ( no se transfieren automáticamente cuando se hace push) se pueden usar los comandos

\begin{verbatim}
git push origin v1.5
git push origin --tags
\end{verbatim}

el primero manda el tag identificado por su etiqueta, y el segundo todos los tags que no se hayan transferido hasta el momento

Entonces, cuando se hace \texttt{clone} o \texttt{pull} desde el repositorio remoto se obtienen todos los tags, junto con el resto de objetos

\end{document}
