\documentclass[]{article}
\usepackage{fontspec}



\begin{document}






\section{Tagging}
Los comandos para ver los repositorios remotos, ver la dirección completa, añadir nuevos repositorios, descargar una rama remota, sin hacer \textit{merge} con la rama master local, 
\begin{verbatim}
git remote 
git remote -v
git remote add pb git://github.com/paulboone/ticgit.git
git fetch pb
\end{verbatim}

\verb= $ git fetch [remote-name] =

El comando trae todos los datos del repositorio remoto . Se dispone de referencias a todas las ramas del repositorio, que se pueden mezclar o inspeccionar en cualquier momento.

Si se hace un \texttt{clone} de un repositorio remoto se añade automáticamente este repositorio con el nombre \textit{origin} Entonces \verb= git fetch origin = \textit{pulls } los datos pero no los mezcla en el repositorio local y no modifica la \texttt{working copy} Se debe hacer \texttt{merge} manualmente

Esta rama no aparece cuando \verb= git branch = pero se puede hacer \texttt{checkout} a ella \verb= git checkout origin/master =

Si se desea obtener información, cambiar el nombre de la referencia al repositorio remoto o borrar dicha referencia se usan los comandos

\begin{verbatim}
git remote show origin

git remote rename pb paul
git remote rm pb

\end{verbatim}

\end{document}
