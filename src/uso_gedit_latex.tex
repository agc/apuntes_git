\documentclass[]{article}
\usepackage{fontspec}
%opening
% rubber: set program xelatex
% rubber: --clean 
% rubber: --into ../
\title{}
\author{}

\begin{document}

\maketitle

\begin{abstract}
Notas sobre como utilizar gedit en la edición de los documentos que componen estos apuntes
\end{abstract}

\section{Introducción}
Se han instalado los plugins de gedit, en particular, 

\begin{verbatim}
sudo yum install gedit-plugins
 sudo yum install gedit-latex
\end{verbatim}
 


La compilación en el plugin gedit se lleva a cabo con rubber. Se ha añadido, en el fichero de texto, el comentario \verb= %rubber: set program xelatex= para que se use xelatex. 

Para que la salida de la compilación vaya al directorio superior he sustituido

\verb= rubber --inplace --maxerr -1 --short --force --warn all --pdf "$filename" %

por 

\begin{verbatim}
 xelatex -output-directory=../ -interaction=nonstopmode "$filename"
\end{verbatim}


La compilación con \verb=Ctrl+Alt+1= es muy rápida. Para abrir el fichero pdf automáticamente se incluye

\begin{verbatim}

gnome-open "$directory/pdf/$shortname.pdf"

\end{verbatim}

pero no funciona(pdf es un enlace simbólico al directorio superior)
\end{document}
